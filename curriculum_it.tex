\documentclass{paper}
\usepackage[utf8]{inputenc}
\usepackage{geometry}
\usepackage{multicol}
\usepackage{hyperref}
\usepackage{xcolor}
\usepackage{enumitem}

\title{Curriculum Vitae}
\author{Leonardo Florenzano}

\geometry{
    a4paper,
    top=5mm,
    left=10mm,
    right=10mm,
    bottom=5mm
}

\hypersetup{
    colorlinks=true,
    urlcolor=blue,
    pdftitle={Leonardo Florenzano - CV}
}

\begin{document}

\maketitle

\begin{multicols}{2}

\noindent
Età: \textcolor{blue}{20}\\
Nazionalità: \textcolor{blue}{Italiana}\\
Residenza: \textcolor{blue}{Bologna, Italia}

\columnbreak

\noindent
Email: \href{mailto:leonardo.florenzano@libero.it}{leonardo.florenzano@libero.it}\\
LinkedIn: \href{https://www.linkedin.com/in/leoflo}{leoflo}\\
Sito: \href{https://leoflo.me}{leoflo.me}

\end{multicols}

\noindent
Sono un appassionato di informatica. Adoro l'architettura del software e il mondo dell'open source.

\section{Esperienza Lavorativa}

\begin{itemize}
    \item \textbf{Internship @ CINECA}: Progetto "LogParser" \textcolor{gray}{(9/2023 - 10/2023)}
    \item \textbf{Internship @ CINECA}: Progetto "Esse3 x Uniroma TranslationLayer" \textcolor{gray}{(2/2023 - 3/2023)}
\end{itemize}

\section{Competenze}

\begin{itemize}
    \item \textbf{Linguaggi di programmazione} che conosco molto bene ed uso attivamente, sia per lavoro, sia per progetti personali:

    Java, C++, Rust, Python, Dart, Javascript, Nix, C.

    \item \textbf{Framework e librerie} usate in progetti lavorativi e personali:

    \begin{itemize}
        \item \textbf{Java}: Springboot, JavaEE (Servlet), Lombok, JNI (Java Native Interface), JDBC, HikariCP
        \item \textbf{C++}: Qt, JNI (Java Native Interface)
        \item \textbf{Dart}: Flutter
        \item \textbf{Javascript}: React
        \item \textbf{Python}: Pygame, Flask, Django
        \item \textbf{Rust}: Actix, hashcash
    \end{itemize}

    \item \textbf{Database} management software che so usare ed integrare in qualsiasi progetto confidentemente:

    \begin{itemize}
        \item \textbf{Relazionali}: MySQL/MariaDB, OracleDB, PostgreSQL, SQLite
        \item \textbf{A documenti}: MongoDB
        \item \textbf{In memoria}: Redis
        \item \textbf{Motori di ricerca}: Elasticsearch
    \end{itemize}

    \item \textbf{Software} che so come funziona in modo dettagliato, configurare, usare ed integrare in qualsiasi progetto sul quale sto lavorando:

    \begin{itemize}
        \item \textbf{Amministrazione di sistemi}: OpenSSH, UWF, Docker, Nix (NixOS and Nix package manager)
        \item \textbf{Server Web}: Apache Web Server, Nginx
        \item \textbf{Gestione del codice sorgente \& DevOps}: Git, Github Actions, Gitlab CE, Gitea, cgit
        \item \textbf{Condivisione file}: rsync, Syncthing, Samba
        \item \textbf{Gestione password}: Keepass, Bitwarden
        \item \textbf{Sistemi operativi}: Windows, Linux
        \item \textbf{Office work}: Microsoft Office, Libreoffice, Latex
    \end{itemize}
\end{itemize}

\section{Titoli}

\begin{itemize}
    \item Cambridge FIRST, livello B2
    \item Primo soccorso aziendale (338/2003)
    \item IRC, abilitazione alla rianimazione cardiopolmonare \textcolor{gray}{(scade a: Maggio 2025)}
\end{itemize}

\section{Esperienze}

\begin{itemize}
    \item OIS: 2021, 2022, 2023, 2024
    \item Olicyber: 2024

    \item IIOT 2022 \textcolor{gray}{(@ IIS Aldini Valeriani, Bologna, Italia)}
    \item \href{https://forum.olinfo.it/t/iiot-2023-diary/8278}{IIOT 2023} \textcolor{gray}{(@ Accademia Araba per la Scienza, la Tecnologia e il Trasporto Marittimo (AAST), Port Fouad, Egitto)}

    \item \href{https://www.ilrestodelcarlino.it/bologna/cronaca/sfida-tra-studenti-sul-supercalcolo-vincono-le-scuole-aldini-valeriani-8638a766}{Contesto di Super Calcolo} \textcolor{gray}{(organizzato da CINECA, in collaborazione con Assemblea regionale Emilia Romagna)}
\end{itemize}

\section{Educazione}

\begin{itemize}
    \item Università di Bologna, Ingegneria Informatica, \textcolor{gray}{2024 - Adesso}
    \item I.I.S Aldini Valeriani, Informatica e Telecomunicazioni, \textcolor{gray}{2019 - 2024, Maturità: 95/100}
\end{itemize}

\end{document}
