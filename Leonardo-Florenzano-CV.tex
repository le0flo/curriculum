\documentclass{paper}
\usepackage[utf8]{inputenc}
\usepackage{geometry}
\usepackage{multicol}
\usepackage{hyperref}
\usepackage{xcolor}
\usepackage{enumitem}

\title{Curriculum Vitae}
\author{Leonardo Florenzano}

\geometry{
    a4paper,
    top=5mm,
    left=10mm,
    right=10mm,
    bottom=5mm
}

\hypersetup{
    colorlinks=true,
    urlcolor=blue,
    pdftitle={Leonardo Florenzano - Curriculum Vitae}
}

\begin{document}

\maketitle

\begin{multicols}{2}

\noindent
Età: \textcolor{blue}{19}\\
Nazionalità: \textcolor{blue}{Italiana}\\
Residenza: \textcolor{blue}{Bologna, Italia}

\columnbreak

\noindent
Email: \href{mailto:leonflorenzano@gmail.com}{leonflorenzano@gmail.com}\\
LinkedIn: \href{https://www.linkedin.com/in/leoflo}{leoflo}\\
Sito: \href{https://leoflo.me}{leoflo.me}

\end{multicols}

\noindent
Sono un appassionato di informatica in tutti i suoi aspetti, dedicato in particolare all'architettura del software.
Mi reputo una persona estremamente ordinata e consapevole di ogni azione, minuta o meno che sia, e delle sue conseguenze.

\section{Esperienza Lavorativa}

\begin{itemize}
    \item \textbf{2023, CINECA, Stage scolastico del 5° anno:}
    
    Nel secondo stage presso CINECA, in compagnia di un altro collega,
    abbiamo sviluppato un programma che permette all'utente di isolare
    informazioni rilevanti all'interno dei log prodotti dal sistema Esse3.
    Questo programma riesce ad individuare, raggruppare e riportare
    le informazioni richieste, tra centinaia di gigabytes di logs in secondi.
    Inoltre implementa un sistema di backtracking delle varie ricerche,
    per aiutare gli investigatori nel loro lavoro.
    Il nostro programma è attivo e in costante utilizzo all'interno dell'azienda.

    \item \textbf{2023, CINECA, Stage scolastico del 4° anno:}
    
    Questa è stata la prima esperienza di lavoro in una software house.
    Ho lavorato insieme ad altri 3 colleghi ad una "translation layer"
    tra l'api dell'università La Sapienza di Roma (uniroma1) e
    il prodotto software offerto da CINECA, Esse3. Il nostro programma
    è attualmente in produzione e svolge attivamente il suo compito.
    
    \item \textbf{2019 - 2021, Volontariato:}
    
    Come primissima esperienza lavorativa, ho occupato la posizione
    di volontario presso il Centro sociale Modiano, a Borgonuovo.
    Un centro per persone con disabilità in cui ho lavorato dal 2019 al 2021.
    In seguito, nel 2022 ho fatto uno stage presso la Pubblica Assistenza
    di Sasso Marconi come Centralinista. L'esperienza è durata 1 mese.
\end{itemize}

\section{Certificazioni e titoli}

\begin{itemize}
    \item Cambridge FIRST, livello B2
    
    \item Primo soccorso aziendale (338/2003)
    
    \item IRC, abilitazione alla rianimazione cardiopolmonare
    e uso del defibrillatore DAE \textcolor{cyan}{(conseguita
    nel 2023, valida per 2 anni, scade a maggio del 2025)}
    
    \item FIN certificazione per assistente bagnanti
    \textcolor{cyan}{(conseguita nel 2023, valida per
    1 anno, scaduta a maggio del 2024)}
\end{itemize}

\section{Educazione}

\begin{itemize}
    \item I.I.S Aldini Valeriani,
    Informatica e Telecomunicazioni,
    \textcolor{cyan}{2019 - 2024,}
    \textcolor{gray}{Maturità: 95/100}

    \item Università di Bologna,
    Ingegneria Informatica,
    \textcolor{cyan}{2024 - Adesso}
\end{itemize}

\section{Competenze informatiche}

\begin{itemize}
    \item \textbf{Linguaggi di programmazione:}
    
    C, C++, Java, Rust, Python, Dart, Javascript,
    Typescript, PHP.

    \item \textbf{Framework e librerie:}
    
    Springboot, JavaEE (Servlet),
    JNI (Java Native Interface), Qt (C++), Flutter,
    React.js, Pygame, Flask, Django, Lombok, JDBC,
    HikariCP, Hashcash.

    \item \textbf{Database:}
    
    MySQL/MariaDB, OracleDB, PostgreSQL, MongoDB, Redis.

    \item \textbf{Software e Tecnologie:}
    
    Apache Web Server, NGINX, Postfix, Dovecot, Samba,
    Teamspeak, OpenSSH, Git, Gitea, Gitlab, Keepass,
    Syncthing, Latex, Linux, Windows, Office (MS e LibreOffice).
\end{itemize}

\section{Soft skills}

\begin{itemize}
    \item \textbf{Relazione con le persone:}
    
    Sono una persona decisamente introversa,
    ma sempre aperta a conversazioni e confronto.
    Ho una ottima pazienza con chiunque mi si presenti
    e sono sempre disponibile ad aiutare quando è possibile.
    Queste caratteristiche si sono sviluppate durante
    il periodo di volontariato.

    \item \textbf{Lavoro individuale:}
    
    Sono molto più a mio agio nel lavorare da solo,
    ma le esperienze a CINECA e in altri ambienti non
    necessariamente lavorativo mi ha insegnato a
    collaborare e coordinarmi con il gruppo.
\end{itemize}

\section{Hobbies}

\begin{itemize}
    \item \textbf{Programmazione:}
    
    Ho iniziato a mettere mano sui linguaggi di
    programmazione all'età di 13 anni e non ho mai smesso di
    programmare da allora. Ho iniziato con Python,
    seguito da C++, inizialmente introdottomi dalle Olimpiadi
    Italiane di Informatica a Squadre (OIS), a cui ho
    regolarmente partecipato in 4 dei 5 anni di scuola
    superiore. Successivamente, tra scuola e l'interesse
    per alcuni videogiochi, ho iniziato ad imparare Java,
    che successivamente diventò presto il mio principale
    linguaggio di programmazione. Nel frattempo, imparai
    il linguaggio Dart, per mettere mano nell'ambito delle
    app per dispositivi mobili. Ho preso dimestichezza con
    C grazie alla mia passione per Linux e il mio interesse
    verso il free software. Questa stessa passione sta spostando
    il mio interesse su Rust e il mondo della cyber security.
    So lavorare con database relazionali come MySQL/MariaDB
    e OracleDB e database non relazionali come MongoDB e Redis.
    Ho una buona conoscenza dei container e tecnologie come
    Docker e Podman, insieme ad una conoscenza superficiale di
    Kubernetes. Ho sempre creato siti e sperimentato con le
    tecnologie web. Ho una ottima conoscenza del CSS, una buona
    esperienza con Javascript e librerie come React.js.

    \item \textbf{Scienze:}
    
    A prescindere dal mio interessamento relativo al mondo
    dell'informatica, ho sempre avuto una passione per il mondo
    della scienze e di tutte le sue sfaccettature. Dalla
    matematica alla fisica, dalla chimica alla biologia, sono
    nato e cresciuto come una persona curiosa, la cui curiosità
    ha trovato conforto nelle materie scientifiche. Non ha caso
    ho scelto Ingegneria Informatica. Il corso mi offre basi
    che vanno oltre alla conoscenza informatica. Mi permette
    di conoscere più a fondo materie come la fisica, con una
    grande attenzione all'elettronica, insieme alla matematica.

    \item \textbf{Musica:}
    
    La musica è una parte fondamentale della persona che sono.
    Sono cresciuto ascoltando musica e suonando strumenti.
    Ho praticato la batteria per 11 anni di fila, partecipando
    ad eventi come il Rockin'1000, suonando per orchestre e
    band. Ho appreso le basi del pianoforte e continuato a
    studiarlo da solo nel tempo libero.

    \item \textbf{Sport:}
    
    Sono sempre stato una persona pressocchè attiva.
    Ho praticato numerosi sport di gruppo e individuali.
    In particolare atletica leggera e orienteering sono
    stati i due sport che più mi hanno intrattenuto e aiutato
    a crescere, sviluppando capacità che altrimenti non avrei
    potuto sviluppare.
\end{itemize}

\end{document}