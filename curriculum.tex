\documentclass{paper}
\usepackage[utf8]{inputenc}
\usepackage{geometry}
\usepackage{multicol}
\usepackage{hyperref}
\usepackage{xcolor}
\usepackage{enumitem}

\title{Curriculum Vitae}
\author{Leonardo Florenzano}

\geometry{
    a4paper,
    top=5mm,
    left=10mm,
    right=10mm,
    bottom=5mm
}

\hypersetup{
    colorlinks=true,
    urlcolor=blue,
    pdftitle={Leonardo Florenzano - CV}
}

\begin{document}

\maketitle

\begin{multicols}{2}

\noindent
Età: \textcolor{blue}{20}\\
Nazionalità: \textcolor{blue}{Italiana}\\
Residenza: \textcolor{blue}{Bologna, Italia}

\columnbreak

\noindent
Email: \href{mailto:leonardo.florenzano@libero.it}{leonardo.florenzano@libero.it}\\
LinkedIn: \href{https://www.linkedin.com/in/leoflo}{leoflo}\\
Sito: \href{https://leoflo.me}{leoflo.me}

\end{multicols}

\noindent
Sono un appassionato di informatica in tutti i suoi aspetti, con una particolare passione verso l'architettura del software e il mondo dell'open source.

\section{Esperienza Lavorativa}

\begin{itemize}
    \item \textbf{2024 - Adesso, Piscina comunale di Sasso Marconi}

    Istruttore di nuoto, Part Time

    \item \textbf{2023, CINECA, Alternanza scuola lavoro del 5° anno:}

    Progetto LogParser.

    \item \textbf{2023, CINECA, Alternanza scuola lavoro del 4° anno:}

    Progetto Esse3 x Uniroma TranslationLayer.

    \item \textbf{2019 - 2021, Volontariato:}

    \begin{itemize}
        \item Centro sociale Saul D. Modiano;
        \item Centralinista presso la Pubblica Assistenza di Sasso Marconi.
    \end{itemize}
\end{itemize}

\section{Certificazioni}

\begin{itemize}
    \item Cambridge FIRST, livello B2

    \item Primo soccorso aziendale (338/2003)

    \item IRC, abilitazione alla rianimazione cardiopolmonare e uso del defibrillatore DAE \textcolor{gray}{(conseguita nel 2023, valida per 2 anni, scade a Maggio del 2025)}

    \item FIN certificazione per assistente bagnanti \textcolor{gray}{(conseguita nel 2023, valida per 1 anno, scaduta a Maggio del 2024)}

    \item FIPSAS istruttore di nuoto, primo livello \textcolor{gray}{(conseguita nel 2023, valida da Febbraio 2024 con scadenza indeterminata)}
\end{itemize}

\section{Educazione}

\begin{itemize}
    \item I.I.S Aldini Valeriani, Informatica e Telecomunicazioni, \textcolor{gray}{2019 - 2024, Maturità: 95/100}

    \item Università di Bologna, Ingegneria Informatica, \textcolor{gray}{2024 - Adesso}
\end{itemize}

\section{Competenze}

In ambito informatico:

\begin{itemize}
    \item \textbf{Linguaggi di programmazione:}

    C, C++, Java, Rust, Python, Dart, Nix, Javascript, Typescript, PHP.

    \item \textbf{Framework e librerie:}

    Springboot, JavaEE (Servlet), JNI (Java Native Interface), Qt, Gtk, Flutter, React.js, Angular, Pygame, Flask, Django, Lombok, JDBC, HikariCP, Hashcash, Actix Web.

    \item \textbf{Database:}

    MySQL/MariaDB, OracleDB, PostgreSQL, MongoDB, Redis, Elasticsearch.

    \item \textbf{Software:}

    Apache Web Server, NGINX, Postfix, Dovecot, Samba, Teamspeak, OpenSSH, Git, cgit, Gitea, Gitlab, Keepass, Syncthing, Rsync, Latex, Linux, NixOS, Nix Package Manager, Windows, Office (MS e LibreOffice).
\end{itemize}

\section{Esperienze}

\begin{itemize}
    \item OIS Edizioni: 2021, 2022, 2023, 2024
    \item IIOT 2022 in Italia \textcolor{gray}{(presso la scuola IIS Aldini Valeriani di Bologna)}
    \item IIOT 2023 in Egitto \textcolor{gray}{(presso l'Accademia Araba per la Scienza, la Tecnologia e il Trasporto Marittimo (AAST) di Port Fouad)}
    \item Olicyber 2024
    \item Contest di calcolo, \textcolor{gray}{organizzato da CINECA in collaborazione con l'assemblea regionale Emilia Romagna}
    \item Presentazione del progetto SAVIA
\end{itemize}

\section{Soft skills}

\begin{itemize}
    \item \textbf{Relazione con le persone:}

    Sono una persona decisamente introversa, ma sempre aperta a conversazioni e confronto.
    Ottima pazienza nei confronti di chiunque mi si presenti.
    Sono sempre disponibile ad aiutare quando è possibile.

    \item \textbf{Lavoro individuale:}

    Sono molto più a mio agio nel lavorare da solo.
    Mi riesco ad adattare facilmente alle dinamiche di lavoro di squadra/gruppo.
\end{itemize}

\end{document}
